% This is the code for the 3I03 Assignment 3 Analyzing Presentations
% Mostafa Okasha - 001304347 - February - 07 - 2017

\documentclass[12pt]{article}

    \setlength{\topmargin}{-0.2in}
	\addtolength{\oddsidemargin}{-.875in}
	\addtolength{\evensidemargin}{-.875in}
	\addtolength{\textwidth}{1.5in}
	\addtolength{\topmargin}{-.875in}
	\addtolength{\textheight}{1.75in}

\title{3I03 Assignment 3: Analyzing Presentations}
\author{Mostafa Okasha - 001304347}
\date{\today}
\begin{document}
\maketitle
\vspace{50mm} %5mm vertical space
\setlength{\parindent}{3.5cm}
\hangindent=3.5cm 
Presentation 1: TED talk: Dophne Koller - Online Education\\ Presentation 2: Linux: Andrew Chatham - Self Driving Cars

\newpage

\section*{Section 1: Analysis}
\subsection*{Analysis of TED Talk: Online Education}
\setlength{\parindent}{1.5cm}
\vspace{5mm} %5mm vertical space

    Professor Daphne Koller, a Standford graduate and founder of the online education platform Coursera, has presented a strong case in her TED Talk in favor of further developing our online education system. Her presentation, which took place at the TED Global conference, was mostly about enlightening the benefits of personalized and well developed online education platforms which are accessible and free to the public. The majority of the audience was highly educated as the requirements for attending this conference was strictly intellect based. Nevertheless, Prof. Koller remained to-the-point, simple and enthusiastic during her presentation and gave out many examples and explanations to support her statements which shows her tailoring her presentation for the audience.
    
\vspace{2.5mm} %5mm vertical space


    The presentation was strictly regarding an improved and more available form of education whereby millions of people around the world, regardless of their circumstances, can have the opportunity to an equal and more efficient educational experience. Prof. Koller starts off by identifying the benefits of the present available online education platform, Coursera. She mentions the improved accessibility, reduced or no price, better instructors offering top-quality education and most importantly, a more interactive and personalized approach of learning. She states that the problem with our current education system is the century old lecture based format which does not ignite the mind's creativity nor imagination. Not only does it lack in personalized and active learning, but is also only offered to those who can afford and access it.
    
\vspace{2.5mm} %5mm vertical space

    The development of the current online education system has been highly reliable on technology according to Prof. Koller. The research performed on the millions of submissions completed online has allowed for constant personalized improvement due to the analysis of all the submissions which allowed for the ability of instant feedback to the students. Overall, Prof. Koller suggests the heavy investment into this educational platform in order to increase it's availability, capacity and performance. And with everyone in the world having access to such high level education, the Einsteins of the world will be able to discover themselves and make this world a better place.

\vspace{2.5mm} %5mm vertical space

    What Prof. Koller lacked in body-language, she made up with her supporting slides and enthusiasm. Her slides would always support her statements by showing either pictures, diagrams, figures or graphs of whatever she was speaking of. They had the perfect amount of detail and whatever they lacked, she explained thoroughly throughout her presentation. Furthermore, the slides did not specifically focus on the main points presented but were rather used to show the benefits of an Online Educational Platform. Her choice of words were very artistic and helped with maintaining focus and interest throughout the presentation.
    
\vspace{2.5mm} %5mm vertical space

    Finally, the structure of her presentation followed a deep introduction of the privileges of a good education and what the lack of it can cause. She suggests the solution of Online Education which offers many benefits in comparison to our regular education system. She then summarizes the importance of such a platform and how it can be used to change the world.
    
\vspace{2.5mm} %5mm vertical space

\newpage

\subsection*{Analysis of Linux Conference: Self Driving Cars}
\vspace{5mm} %5mm vertical space

    Andrew Chatham, a principle Engineer at Google and a Computer Science graduate from Duke University, presented at the Embedded Linux Conference in 2013. The presentations presented during this event were targeted towards individuals who have been in the technology business for years and came to listen to the main Keynotes of several business's endeavor towards certain milestones of technology. Although Chatham started off his presentation with an introduction to the importance and risks of driving along with the current issues our society faces with automobiles, he later goes into detail about concepts used in creating a functional self-driving car. The presentation was obviously tailored towards the audience but not all technology enthusiasts are aware of certain technologies used in the autonomous world so perhaps a little clarification of terms would have made things a lot more clear.
    
\vspace{2.5mm} %5mm vertical space

    As mentioned before, Chatham's presentation started off with an introduction to the importance of driving and the issues we face daily from transportation including accidents, being stuck in traffic and spending a lot time focusing on a task that does not require much critical thinking. He discusses how human error is the main cause of accidents globally and that with the concept of automated driving, this issue could be resolved. As Chatham has been working with Google on automated driving for 10 years, he then discusses how they have developed over the years and how they currently work. His slides were initially structured to compliment his statements with excellent visuals to aid his examples and explanations whenever he would mention anything. They were structured to his presentation and started with the introduction to the current problems they are facing while working at Google. They then discuss how the vehicle is built starting from the sensors used, how the system finds other cars, how the vehicle perceives itself in the world, the car's movement planning, interacting with the environment/humans, being aware of driving rules/regulations and aware of other nearby vehicles.

\vspace{2.5mm} %5mm vertical space

    Although the presentation was more about informing the audience of the importance of automated driving and to essentially increase awareness of the current technology Google possesses on this topic, Chatham mentions that the main laser and radar systems used in these cars can be affected by weather conditions which in turn cause a disruption to the data being fed to the control system. Sadly however, he does not propose a solution to the problem they are currently facing as that is something that is not entirely in the control of the Self-Driving Car team at Google. Thus the question remained, how can we detect unusual situations while driving autonomously and handle the driving back to the user in case faults occur.
 
\vspace{2.5mm} %5mm vertical space

    Finally, the slides presented by Chatham were extremely well prepared. They always demonstrated what he was talking about in simple words and diagrams that would not need the audience to lose focus with the actual presentation to be read. Though what distinguished this presentation and made it exceptional were the videos accompanied by Chatham's explanation on how the Autonomous car functions behind the scenes. They fit extremely well with his explanation and along with his great body language, intriguing questions and excellent choice of words move this presentation to the top.

\newpage

\section*{Section 2: Compare and Contrast}

    Overall, Chatham's presentation was better than Koller's in terms of context and information.  And Although Koller had an edge over Chatham in presentation skills and preparation, Chatham's excitement about the topic toppled everything over to his side as the audience was more intrigued and focused with the presentation than in Koller's. The interest of the topic at hand and the reason why the presentation held place has definitely played a significant difference in this comparison which is why it is preferable to state the strengths and weakness of both presentation since a comparison would deem less useful.
    
\subsubsection*{Strengths and Weaknesses of TED Talk:}

\begin{itemize}
\item (+) Prof. Koller presents strongly and seriously with the right touch of comedy added to the presentation. The Short video and slides greatly support the material being discussed in the presentation and make it a lot easier to follow along.
\item ( - ) Nevertheless, she does not suggest a  solution to the problem she indicated at the beginning of her presentation along with also stating the already obvious solution without giving more detail. The summary of the presentation she performs at the end does not inform the audience of a single key point that should be remembered from the presentation, she goes on to list several items instead.
\end{itemize}

\subsubsection*{Strengths and Weaknesses of Linux Presentation:}

\begin{itemize}
\item (+) Chatham did an excellent job at repeating the most important points of the presentation more than once. Furthermore, he used a lot of body language and spoke with enthusiasm about the topic which causes the audience to gain interest in his presentation.
\item ( - ) However, he talks about very specific complex terms such as the "SCHED FIFO" without giving an explanation of what they are at the very least. He also never interacted with the audience which could have further increased their interest and concentration during the presentation.
\end{itemize}

\subsubsection*{Improvements for TED Talk:}
    Koller's presentation could have been improved if she talked in more detail about her suggested solution instead of just stating facts as this would in turn allow the audience to see the vision she intends to take with her Problem. Furthermore, she should also amend her supporting slides to focus more on the key points of the presentation rather than the random facts she talks about.

\subsubsection*{Improvements for Linux Conference:}
    Although Chatham's presentation was excellent, he did not interact with the audience which would have otherwise improved his presentation even more since it would get the audience involved and focused. Another improvement would be to summarize the key points of his presentation near the end as to leave the audience with the single most important idea in regards to the presentation.

\end{document}